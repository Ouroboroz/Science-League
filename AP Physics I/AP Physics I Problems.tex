\documentclass{article}
\usepackage{amsmath}
\usepackage{pgfplots}
\pgfplotsset{width=3in,compat=1.9}
\usepackage[inline]{enumitem}  
\usepackage[letterpaper, total={7.5in, 10in}]{geometry}
\title{JPS Science League: AP Physics I}
\author{Kevin Yang and Elisha Zhao}
\date{Entrance Exam}
\makeatletter
% This command ignores the optional argument for itemize and enumerate lists
\newcommand{\inlineitem}[1][]{%
\ifnum\enit@type=\tw@
    {\descriptionlabel{#1}}
  \hspace{\labelsep}%
\else
  \ifnum\enit@type=\z@
       \refstepcounter{\@listctr}\fi
    \quad\@itemlabel\hspace{\labelsep}%
\fi}
\makeatother
\parindent=0pt
\begin{document}
	\maketitle
	\renewcommand{\labelenumii}{\alph{enumii})}
	\textbf{Instruction: }There are $25$ test questions, $5$ bonus questions and $1$ thought provoking question in this exam. The $25$ questions will determine your placement while the $5$ bonus questions will serve as tie breakers. The thought provoking question is just for you to think about after the exam if you enjoy physics. You will be given $50$ minutes for this exam. Points will not be taken off for wrong answers so you are encouraged to answer every question. Remember, finish the $25$ questions before starting the bonus. Suppose that $g=10\;\frac{m}{s^2}$. Good luck and have fun!
	\begin{enumerate}
		\item[1)]  %GRAPH
		To get to school everyday, Kevin must wake up really early. Normally he wakes up an hour before the time he has to leave but he always snoozes until the last moment. On the day Kevin had to give out the entrance exam, Kevin accidentally snoozed one too many times and is extremely behind schedule. To make up for the lost time, Kevin started up his car and began accelerating in a straight line at $4\;\frac{m}{s^2}$. Which of the following could be a graph of his motion?
		\begin{enumerate}
			\item 
			\inlineitem 
			\inlineitem 
			\inlineitem
		\end{enumerate}
		%Question 2: Projectile Problem
		\item[2)]  A major issue in the news is the Trump border wall. Suppose that such a wall were to be build and the wall is $11\;m$ tall. Kevin is a very curious kid and wants to see at what angle would it take to launch a ball over the wall. Sadly, Kevin isn't very bright and built a cemented down launcher located at $14\;m$ away from the wall. Furthermore, the launcher can only shoot at an angle of $56$ degrees. What is the minimum speed so the ball can just pass over the top of the wall.
		\begin{enumerate}
			\item
			\inlineitem 
			\inlineitem 
			\inlineitem
		\end{enumerate}
	\end{enumerate}
	%Ramp on a building and then the ball rolls off the ramp and then off the building.
	\textbf{Use the following information for Questions \#3 and \#4:} Kevin got bored one day and accidentally built a $500\;m$ tall building. On top of the building, Kevin added a ramp with an incline of $30$ degrees and height of $10\;m$. The ramp does not extend the entire width of the building such that the end of the ramp is not the edge of the building(the ball has to roll on a flat surface after leaving the ramp). Suppose that all transitions are smooth and all forces of friction are negligible since Kevin is so great at engineering.
	\begin{enumerate}
		%Question 3: Object rolling off a ramp
		\item[3)] One day, Kevin stole a bowling ball from the bowling ball team. After being chased to the top of his building, Kevin decided that the only way out was by rolling the bowling ball off the roof. In order to give the bowling ball, which as a mass of $7\;kg$, enough velocity, Kevin decided to roll the ball off the ramp. After placing the bowling ball at the top of the ramp, what is the velocity of the ball at the bottom of the ramp?
		\begin{enumerate}
		 %Change the assumed velocity in following problem to trick the kiddos
			\item
			\inlineitem 
			\inlineitem 
			\inlineitem
		\end{enumerate}
		%Question 4: Object rolling off a building
		\item[4)] Regardless of your answer to the last problem, assume that velocity at the end of the ramp is $20\;\frac{m}{s}$. Determine the horizontal distance the bowling ball could travel after it leaves the roof. 
		\begin{enumerate}
			\item
			\inlineitem 
			\inlineitem 
			\inlineitem
		\end{enumerate}
		%Question 5: Force Analysis on Rolling Ball
		\item[5)] Suppose there is a ball with a mass of $10\;kg$. Let the ball be rolling without sliding on a rough surface at $3\;\frac{m}{s}$. You are given the static and kinetic coefficients of friction to be $\mu_s = 0.67$ and $\mu_k = 0.3$. Find the amount of force needed to be exerted at a $30$ degree angle to keep the ball moving at a constant velocity.
		\begin{enumerate}
			\item
			\inlineitem 
			\inlineitem 
			\inlineitem
		\end{enumerate}
		%Question 6: Atwood Machine
		\item[6)] A really common physics problem given to beginning physics students is called the Atwood machine. This setup is so commmonplace in the world of physics education that even the US Physics Team uses it in some form. An Atwood machine is basically two weights connected by a string draped over a pulley. The pulley in this case is completely massless and frictionless. The mass on the left side is $12.6\;kg$ while the mass on the right side is $3.7\;kg$. What is the tension on the string connecting the two masses?
		\begin{enumerate}
			\item
			\inlineitem 
			\inlineitem 
			\inlineitem
		\end{enumerate}
	\end{enumerate}
	\textbf{Use the following information for Questions \#7 and \#8: }There is a ramp of $45$ degrees from the ground and coefficients of friction $\mu_s = 0.62$ and $\mu_k = 0.47$. A box with a mass of $14\;kg$ is placed on the ramp.
	\begin{enumerate}
		%Question 7: Person pushing box on ramp
		\item[7)] Elisha can only exert a force horizontal to the ground. Find the amount of force she needs to exert inorder to keep the box in static equilibrium.
		\begin{enumerate}
			\item
			\inlineitem 
			\inlineitem 
			\inlineitem
		\end{enumerate}
		%Problem 8: Free sliding box on ramp
		\item[8)] If Elisha were to stop pushing the box. Find the acceleration the box would have as it slides down the ramp.
		\begin{enumerate}
			\item
			\inlineitem 
			\inlineitem 
			\inlineitem
		\end{enumerate}
		%Question 9: Velocity after rolling off building
		\item[9)]
		When Kevin was writing this test, he kinda got bored and decided to build something. Again, he accidentally built a building but this time it is $H\;m$ tall. To express his rage at his ability to accidentally build skyscrapers, Kevin decided to roll a ball off the roof at a horizontal velocity of $v\;\frac{m}{s}$. Kevin chose one of his heaviest balls-weighing at $m\;kg$. However, as a aspiring physicist, Kevin wanted to know what was the final velocity of the ball just before it hit the ground. Can you help him figure it out?
		\begin{enumerate}
			\item
			\inlineitem 
			\inlineitem 
			\inlineitem
		\end{enumerate}
		\item[10)] %Graph Analysis
		Elisha has a metal sphere of $14\;kg$. The sphere is extremely smooth so all forces of friction is negligible. Elisha pushes on the sphere based on the following graph:
		\begin{center}
			\begin{tikzpicture}
				\begin{axis}[
	           		axis lines = left,
					title={Graph of a Metal Sphere},
	    			xlabel= {$t$},
	    			ylabel= {$x$},
	                xmax = 35,
	    			]
					\addplot[
						domain = 0:30,
						color = blue,
					]
					{5*x};
				\end{axis}
			\end{tikzpicture}
		\end{center}
		What is the total kinetic energy of the sphere?
		\begin{enumerate}
			\item
			\inlineitem 
			\inlineitem 
			\inlineitem
		\end{enumerate}
		%Force Analysis: Force between two blocks
		\item[11)]
		Suppose there are two blocks placed right next to each other. The left block is much bigger than the right block with a mass of $30\;kg$. The right block has a mass of $14\;kg$. Let us call the force between the two blocks $F_L$ when a force to the right is exerted on the left block. Let us call the force between the two blocks $F_R$ when a force to the left is exerted on the right block. These two forces are applied separately. How does $F_L$ and $F_R$ compare?
		\begin{enumerate}
			\item
			\inlineitem 
			\inlineitem 
			\inlineitem
		\end{enumerate}
		%Force Analysis: 2 string hanging ball
		\item[12)]
		Elisha is putting up some lamps in order to give the test takers some more light. Each lamp is held in place by $2$ strings. $1$ string attaches to ceiling at an angle of $60$ degrees while the second string perpendicularly attaches to the wall. Each lamp has a weight of $5.5\;kg$. Find the tension in the horizontal string.
		\begin{enumerate}
			\item
			\inlineitem 
			\inlineitem 
			\inlineitem
		\end{enumerate}
		%Question 13: Energy Question
		\item[13)]
		After a lot of persuasion from his friends, Kevin finally went to a therapy course for those who accidentally build skyscrapers. Sadly, along the way, Kevin accidentally build a rollercoaster with two hills. The first hill has a height of $326\; m$ while the second hill has $211\;m$. Sadly, Kevin was a bit tired while he was building the rollercoaster so friction is no longer negligible. Luckily, along the $200\;m$ long path from the top of the first hill to the second, the force of friction is constant at $9\;N$. Can you find the velocity of the car at the second hill if the rollercoaster has a mass of $1000\;kg$?
		\begin{enumerate}
			\item
			\inlineitem 
			\inlineitem 
			\inlineitem
		\end{enumerate}
		%Planetary Motion
		\item[14)]
		Recently, Elisha had discovered a new solar system while looking out of her telescope. She saw that the solar system composed of a star and $3$ planets. The mass of the star is much larger than that of the planets. After some intense calculations, Elisha realized that the sun had a mass of $2*10^30\;kg$. After a few days of observation, she realized that one of the planets with a mass of $3*10^{26}\;kg$ takes $300$ days to make one orbit around the star but she was not able to discern the radius of the orbit. For another planet of mass $8*10^{25}\;kg$ orbiting the star, Elisha realized that the radius of orbit was $2*10^11\;km$ but could not find out the period of orbit. Can you help her calculate this?
		\begin{enumerate}
			\item
			\inlineitem 
			\inlineitem 
			\inlineitem
		\end{enumerate}
	\end{enumerate}
	\textbf{Use the following information for Questions \#15 and \#16: }At his therapy sessions, Kevin realized the importance of responsibility. Since he had built so many buildings, he needed to also create ways of getting people down from the top of the building. A great idea that he came up with is putting a huge spring at the bottom of his building. In order to test this theory out, Kevin decided to make a smaller scale model with a ideal Hookean spring of no mass and a spring constant of $7\;\frac{N}{m}$. The spring is as long as it needs to be. Kevin only has a mass of $15\;kg$
	\begin{enumerate}
		%Question 15: Block dropped onto Spring
		\item[15)]
		 Initially, Kevin drops the mass from a height of $4\;m$ directly onto spring. Find how much the spring would compress by at the point of maximum compression.
		\begin{enumerate}
			\item
			\inlineitem 
			\inlineitem 
			\inlineitem
		\end{enumerate}
		%Question 16: Block dropped onto Spring+Mass
		\item[16)]
		After further planning, Kevin realized that he has to put a pad in order to catch the person. In order to test this, Kevin simply modified his model by adding a mass onto the top of his ideal spring. The new mass is directly attached onto the spring with a mass of $5\;kg$. If the original mass of $15\;kg$ was dropped again from a height of $4\;m$, find the amount the spring would compress by at its maximum compression.
		\begin{enumerate}
			\item
			\inlineitem 
			\inlineitem 
			\inlineitem
		\end{enumerate}
	\end{enumerate}
	\textbf{Use the following information for Questions \#17 and \#18: }Elisha decided to model one of the planets that she just discovered. To do so, she spun a mass around using a string. The string is always horizontal to the ground and the mass makes a perfect circle around Elisha. The mass weights $12\;kg$ and Elisha spins it at $60\;rpm$ at a distance of $2\;m$ away from her. All forms of friction is negligible.
	\begin{enumerate}     
		%Question 17: Rotational: Find centripetal force
		\item[17)]
		Can you help find the amount of force Elisha has to exert in order to provide enough tension on the string so the mass can be held in circular motion?
		\begin{enumerate}
			\item
			\inlineitem 
			\inlineitem 
			\inlineitem
		\end{enumerate}
		%Changing aspects of circular motion+calculating velocity
		\item[18)]
		If Elisha were to decrease the radius of the circle by half and keep the tension on the string the same, at what velocity would the ball fly out once Elisha lets go.
		\begin{enumerate}
			\item
			\inlineitem 
			\inlineitem 
			\inlineitem
		\end{enumerate}
		%Question 19: Bumper Cars
		\item[19)]
		Kevin hates amusements parks. The only ride he ever goes on is the bumper cars. Suppose that there are three bumper cars of different masses lined up. From left to right, the bumper cars are numbered $1$, $2$ and $3$. Car $1$ has a mass of $200\;kg$, Car $2$ has a mass of $325\;kg$ and Car $3$ has a mass of $500\;kg$. Collisions between Car $1$ and Car $2$ are completely elastic. Collisions between Car $2$ and Car $3$ are completely inelastic. If Kevin started up Car $1$ with a velocity of $8\;\frac{m}{s}$ and collided with Car $2$, what velocity would Car $3$ have?
		\begin{enumerate}
			\item
			\inlineitem 
			\inlineitem 
			\inlineitem
		\end{enumerate}
		%Question 20: Pendulum Question
		\item[20)]
		Kevin is often extremely hyperactive and probably has ADHD. In order to calm him down, many of his friends has to use a pendulum to hypnotize him. To create a pendulum, Kevin's friends hung a mass of $2\;kg$ by a string of length $32\;cm$. However, in order to avoid being hypnotized, Kevin simply times his eyes with the pendulum's motion. To do so, Kevin needs to know how long the pendulum takes in order to make one revolution. Suppose that everything is ideal and friction is negligible. Can you help find out how long is the period of the pendulum is?\begin{enumerate}
			\item
			\inlineitem 
			\inlineitem 
			\inlineitem
		\end{enumerate}
		%Question 21: Simple Harmonic-Oscillating Spring
		\item[21)]
		I'm getting really tired at writing creative stories for these problems. $>$ implying the stories are creative. Long story short, there is a spring with a spring constant of $9.4\;\frac{N}{m}$ and there is a mass of $0.46\;kg$ attached onto it. The spring and mass is horizontal on a table with no friction. The rest length of the mass/spring system is $24\;cm$. The spring is then stretched so it now becomes $1.3\;m$. Find the ratio between the maximum force on the mass and the angular velocity of the spring and mass: $\frac{F}{\omega}$.
		\begin{enumerate}
			\item
			\inlineitem 
			\inlineitem 
			\inlineitem
		\end{enumerate}
		%Question 22: Car going up ramp
		\item[22)]
		Kevin just got his examination permit for his car over the summer. To test out his newfound driving abilities, Kevin decided to drive his car up a road. The road that Kevin chose had an incline angle of $30$ degrees above horizontal and has an extremely large coefficient of friction. Kevin's car, weighing a mass of $1225\;kg$, can only output a power of $450,000\;W$. What is the maximum speed that Kevin can achieve on the road?
		\begin{enumerate}
			\item
			\inlineitem 
			\inlineitem 
			\inlineitem
		\end{enumerate}
		%Question 23: Momentum
		Recently, NASA is working on the James Webb Telescope. The JWT will allow NASA to see deep into space with infrared rays. In order to control which direction the telescope is pointed in, NASA uses a very large fly wheel. The point of the flywheel is for you to decide but you shall do some calculations about it anyways. The Telescope has a mass of $6500\;kg$ and the fly wheel has a separate mass of $1000\;kg$. The flywheel was originally spinning at an angular velocity of $2\;rpm$ clockwise. if the flywheel was suddenly flipped $180$ degrees(aka it is now spinning counterclockwise but as the same speed), how fast would the JWT spin and in what direction.
		\begin{enumerate}
			\item
			\inlineitem 
			\inlineitem 
			\inlineitem
		\end{enumerate}
		%Question 24: Bullet into Pendulum.
		\item[24)]
		I have seriously given up being creative. There is a pendulum with a length of $2\;m$ and a mass of $4.3\;kg$ at the bottom. Kevin holds a BB-gun and shoots a bullet into the pendulum. The BB-gun uses a spring with a constant of $812\;\frac{N}{m}$ compressed $4\;cm$ to launch the bullet. The bullet has a mass of $1\;g$. If Kevin fires the BB-gun at the pendulum, what is the maximum height the pendulum can achieve?
		 \begin{enumerate}
			\item
			\inlineitem 
			\inlineitem 
			\inlineitem
		\end{enumerate}
		%Question 25: Projectile same angle
		\item[25)]
		In an act of commemoration, Kevin brought back his ball launcher. This time Kevin made it so one can change the angle of his launcher. At what angle would Kevin have to fire a ball at so it lands the same distance as Kevin firing the ball at $20$ degrees?
		\begin{enumerate}
			\item
			\inlineitem 
			\inlineitem 
			\inlineitem
		\end{enumerate}
	\end{enumerate}

	%EXTRA PAGE
	\newpage
	\textbf{Reminder: } Remember, finish the $25$ actual questions before moving onto the bonus and thought provoking question(TPQ). You will not get any extra time to do these. It is more important for you to place onto the team rather than having tiebreaker points. 
	%Bonuses and TPQ
	\begin{enumerate}
		%DC Circuit
		\item[B1)]
		You are given the following circuit. The battery on the left has a voltage difference of $12\;V$ while the battery on the battery on the right has a \emph{emf} of $23\;V$. $R_1=4\;\Omega$, $R_2=13\;\Omega$, $R_3=7\;\Omega$ and $R_4=17\;\Omega$. Find the power produce by bulb $3$ at any given point of time.
		\begin{enumerate}
			\item
			\inlineitem 
			\inlineitem 
			\inlineitem
		\end{enumerate}
	\end{enumerate}
	\textbf{Use the following information for Questions \#B2 and \#B3: }Suppose that there exists a square with fixed metal orbs on each corner. Each side of the square is $1.2\;m$ long. Let the top left orb be called Orb $1$ with a charge of $3\;C$. Let the top right orb be called Orb $2$ with a charge of $5\;C$. Let the bottom left orb be called Orb $3$ with a charge of $-2\;C$. Let the bottom right orb be called Orb $4$ with a charge of $10\;C$. A neutral non-conducting orb, called Orb $5$, is placed in the center of the square.
	\begin{enumerate}
		%Electic Charges
		\item[B2)]
		What is the voltage between Orb $5$ and Orb $2$?
		\begin{enumerate}
			\item
			\inlineitem 
			\inlineitem 
			\inlineitem
		\end{enumerate}
		%Charges
		\item[B3)]
		If Orb $5$ was to be given a charge of $3.4\;C$. What is the net force on Orb $5$?
		\begin{enumerate}
			\item
			\inlineitem 
			\inlineitem 
			\inlineitem
		\end{enumerate}
		%Waves
		\item[B4)]
		Whose/What principle states that every point of a wave front can be thought of as the center of a new wave of the same frequency has the original wave?
		\begin{enumerate}
			\item
			\inlineitem 
			\inlineitem 
			\inlineitem
		\end{enumerate}
		%Waves
		\item[B5)]
		Kevin wants to build a piano. Kevin knows nothing about a piano except that he should not allow the $7$th harmonic to exist. If Kevin uses a string of length $2.4\;m$ to create the sound, where should he make the hammer of the piano strike so the $7$th harmonic is eliminated?
		\begin{enumerate}
			\item
			\inlineitem 
			\inlineitem 
			\inlineitem
		\end{enumerate}
		\item[TPQ: ]
		Suppose you have a mass $m$ attached to a spring. Let the spring have an original coefficient of $k$. Originally, the mass is set into oscillation with an angular velocity of $\omega$ and a max amplitude of $A_0$. Over a course of $10^6$ years, aka it takes a long time, the spring constant decays to $\frac{k}{2}$ while the mass is still oscillating. What multiple of the original amplitude $A_0$ is the new amplitude $A_f$?
	\end{enumerate}
\end{document}

