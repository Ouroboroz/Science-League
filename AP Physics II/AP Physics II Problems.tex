\documentclass{article}
\usepackage{amsmath}
\usepackage{framed}
\usepackage[inline]{enumitem}  
\usepackage[pdftex]{graphicx}
\usepackage[letterpaper, total={7.5in, 10in}]{geometry}
\title{JPS Science League: AP Physics II}
\author{Kevin Yang and Elisha Zhao}
\date{Entrance Exam}
\makeatletter
% This command ignores the optional argument for itemize and enumerate lists
\newcommand{\inlineitem}[1][]{%
\ifnum\enit@type=\tw@
    {\descriptionlabel{#1}}
  \hspace{\labelsep}%
\else
  \ifnum\enit@type=\z@
       \refstepcounter{\@listctr}\fi
    \quad\@itemlabel\hspace{\labelsep}%
\fi}
\makeatother
\parindent=0pt
\begin{document}
	\maketitle
	\renewcommand{\labelenumii}{\alph{enumii})}
	\textbf{Instructions: }There are $25$ test questions and $5$ bonus questions in this exam. The $25$ questions will determine your placement while the $5$ bonus questions will serve as tie breakers. You will be given $50$ minutes for this exam. Points will not be taken off for wrong answers so you are encouraged to answer every question. Remember, finish the $25$ questions before starting the bonus. Suppose that $g=10\;\frac{m}{s^2}$. Good luck and have fun!
	\begin{enumerate}
		%Question 1: Understanding the meaning of PV diagram // Change?
		\item[1)]
		Kevin loves PV diagrams. Almost as much as he loves the Carnot cycle. Given the following PV diagram:
		%Add diagram

		Which of the following transition type is it:
		\begin{enumerate}
			\item
			\inlineitem 
			\inlineitem 
			\inlineitem
		\end{enumerate}
		%Question 2: PV Diagram - Carnot Cycle
		\item[2)]
		Kevin really loves Carnot cycles. In order to share his love of Carnot cycles with you, he decided to give you a Carnot cycle question. However, a Carnot cycle would be too simple so he decided to spice things up by changing by transition of the Carnot cycle. According to the diagram below:
		%Add diagram

		What is the maximum amount of work that an engine running the proposed cycle can provide?
		\begin{enumerate}
			\item
			\inlineitem 
			\inlineitem 
			\inlineitem
		\end{enumerate}
	\end{enumerate}
	\textbf{Use the following information for Questions \#3 and \#4: }When Kevin was first learning thermodynamics, he was forced to memorize the following chart:
	%Chart of heat
	To share his pain, Kevin is making you fill out some missing parts of the chart. Nevertheless, it is a very important chart and Kevin suggests that you memorize it. Its not that bad.
	\begin{enumerate}
		%Question 3: Completing the heat chart
		\item[3)]
		On the chart, there is a missing square labled $(A)$. Please use the already filling in information of the chart and prior knowledge to deduce what belongs in $(A)$.
		\begin{enumerate}
			\item
			\inlineitem 
			\inlineitem 
			\inlineitem
		\end{enumerate}
		%Question 4: Completing the heat chart
		\item[4)]
		Now that you have completed $(A)$, naturally $(B)$ comes next. Kevin promises that you don't need the answer of $(A)$ to find out what $(B)$ is. Anyways, what belongs in $(B)$.
		\begin{enumerate}
			\item
			\inlineitem 
			\inlineitem 
			\inlineitem
		\end{enumerate}
		%Question 5: Third Law of Thermodynamics
		\item[5)]
		Kevin has a friend who loves Trump. In honor of that friend, here goes this problem. As you know, President Trump wishes to build a wall on our border. Normally, temperatures in those states go up to as much as $115^o$F. Partisan opinions aside, the proposed wall has entropy, in fact, a lot of entropy. Which of the following objects would have the least entropy?
		\begin{enumerate}
			\item An sphere at $0\;K$
			\inlineitem  A crystal at $0\;K$
			\inlineitem  A sphere at infinite $K$
			\inlineitem	 A box at room temperature
		\end{enumerate}
		%Question 6: Ideal gas law
		\item[6)]
		Elisha was working with ideal gases for a science experiment that she was designing. More specifically, she used a monotomic ideal gas-meaning that there is only one atom per molecule. The gas was initially at a temperature of $23^o$C, pressure of $2.3\;atm$, and a volume of $2.2\;L$. Elisha raises the temperature to $45^o$C and allows the pressure to decrease to $0.9\;atm$. What is the new volume of the gas?
		\begin{enumerate}
			\item
			\inlineitem 
			\inlineitem 
			\inlineitem
		\end{enumerate}
		%Question 7: Kinetic Theory
		\item[7)]
		What even is kinetic theory
		\begin{enumerate}
			\item
			\inlineitem 
			\inlineitem 
			\inlineitem
		\end{enumerate}
	\end{enumerate}
	\textbf{Use the following information for Questions \#8 and \#9: }After reading the problems that he wrote, Kevin realized that he is quite narcissistic for using his himself has the main character in all but one question so far. Thus, Kevin wishes to wash these sins off using water. To do so, Kevin designs a shower system. All of the water comes from a water tank placed $20\;m$ into the air. A pipe, placed perpendicular to the ground, with a diameter of $20\;cm$ brings the water down to Kevin's head level of $2\;m$. Before water arrives at the shower head, the $20\;cm$ pipe smoothly becomes a pipe with a diameter of $10\;cm$. Suppose that every component is frictionless, all curves are completely smooth and curved so no energy will be lost and the viscosity of water is neglibible.
	\begin{enumerate}
		\item[8)]
	\end{enumerate}
\end{document}