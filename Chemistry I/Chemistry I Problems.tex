\documentclass{article}
\usepackage{amsmath}
\usepackage{framed}
\usepackage[inline]{enumitem}  
\usepackage[pdftex]{graphicx}
\usepackage[letterpaper, total={7.5in, 10in}]{geometry}
\title{JPS Science League: Chemistry I}
\author{Kevin Yang and Elisha Zhao}
\date{Entrance Exam}
\makeatletter
% This command ignores the optional argument for itemize and enumerate lists
\newcommand{\inlineitem}[1][]{%
\ifnum\enit@type=\tw@
    {\descriptionlabel{#1}}
  \hspace{\labelsep}%
\else
  \ifnum\enit@type=\z@
       \refstepcounter{\@listctr}\fi
    \quad\@itemlabel\hspace{\labelsep}%
\fi}
\makeatother
\parindent=0pt
\begin{document}
	\maketitle
	\renewcommand{\labelenumii}{\alph{enumii})}
	\textbf{Instructions: }There are $25$ test questions that will determine your placement. You will be given $50$ minutes for this exam. Points will not be taken off for wrong answers so you aßre encouraged to answer every question. Good luck and have fun!
	\begin{enumerate}
		%Question 1: Measurement
		\item[1)]
		Something really small can be very important. The prefix "pico" is ued to indicate something very small. What power of $10$ is equivalent to "pico".
		\begin{enumerate}
			\item $-3$
			\inlineitem $-9$
			\inlineitem $-12$
			\inlineitem $-15$
		\end{enumerate}
		%Question 2: Properties
		\item[2)]
		Kevin found the following statements on a piece of crumpled paper lying on the ground.
		\begin{enumerate}
			\item[A)] Mass is conserved
			\item[B)] Moles are conserved
			\item[C)] Volume is conserved
			\item[D)] Molecules are conserved
		\end{enumerate}
		Which is (are) always true for a chemical reaction?
		\begin{enumerate}
			\item A only
			\inlineitem B only
			\inlineitem A and B
			\inlineitem B and D only
		\end{enumerate}
		%Question 3: Compounds
		\item[3)]
		Elisha was studying Chem Lab for Science Olympiad by attempting to match different compounds with their names. Can you help her figure out which compound has the correct name paired with it?
		\begin{enumerate}
			\item Sodium Dioxide and $NO_2$
			\inlineitem Nitrogen Oxide and $NO_2$
			\inlineitem Ozone and $O_3$
			\inlineitem Carbon Bioxide and $CO_2$
		\end{enumerate}
		%Problem 4: Formulas
		\item[4)]
		Kevin often gets hungry at midnight. Often times, he goes to the fridge to find something tasty to eat. 2 days ago, he picked up some calcium nitrate for a small snacc. Can you help him figure out what the chemical formula for calcium nitrate?
		\begin{enumerate}
			\item $CaNO_3$
			\inlineitem $Ca(NO_3)_2$
			\inlineitem $CaNO$
			\inlineitem $CaNO_2)_2$
		\end{enumerate}
		%Problem 5: Formulas
		\item[5)]
		After eating some calcium nitrate, Kevin thought it wasn't very tasty so he decided to try out sodium sulfate. Can you figure out the ratio of the masses of sodium and sulfur in sodium sulfate?
		\begin{enumerate}
			\item $1.52$
			\inlineitem $1.25$
			\inlineitem $0.86$
			\inlineitem $1.43$
		\end{enumerate}
		%Problem 6: Mole
		\item[6)]
		Elisha found a bucket of glucose sugar in Mr. D room. (Don't ask why that bucket exists) Can you find out how many moles of H would you find in $5.2\; kg$ of $C_6H_{12}O_6$.
		\begin{enumerate}
			\item $243.2$
			\inlineitem $346.4$
			\inlineitem $28.9$
			\inlineitem $173.2$
		\end{enumerate}
		%Problem 7: Weight Percent
		\item[7)]
		It turns out that Kevin placed that bucket in Mr. D's room. We don't actually know why he did it but after we asked him, he informed us that he also made a bucket of sucrose($C_{12}H_{22}O_{12}$). Kevin wasn't exactly sure how much sucrose he put in the bucket but he was sure that he put $3.6\;kg$ of carbon atoms. Can you help him figure out how many moles of sucrose he has?
		\begin{enumerate}
			\item $15$
			\inlineitem $25$
			\inlineitem $30$
			\inlineitem $300$
		\end{enumerate}
		%Problem 8: Properties
		\item[8)]
		The US National Chemistry Olympiad is something extremely fun but not exactly easy. While taking the USNCO, Kevin had to perform an experiment to determine the molarity of acid in a beaker. To do so, he decided to titrate 5 individual samples of acid from the beaker. As a result he found the molarity is: $4.53 \pm 0.002$, $2.45 \pm 0.005$, $8.43 \pm 0.001$, $5.23\pm 0.002$, and $6.45\pm 0.001$. Is Kevin accurate and/or precise?
		\begin{enumerate}
			\item Accurate and Precise
			\inlineitem Accurate, Not Precise
			\inlineitem Precise, Not Accurate
			\inlineitem Not Accurate, Not Precise
		\end{enumerate}
		%Problem 9: Compounds
		\item[9)]
		Every day when you're walking down the street, everybody that you meet has an original point of view. And I say - Hey! What a wonderful kind of day! If we could learn to work and play and get along with each other to analyze some compounds with polyatomic ions. One day when Elisha was walking down the street, she found some barium dicrhomate $BaCr_2O_7$. Can you help her figure out how many sodium ions would be in sodium dichromate?
		\begin{enumerate}
			\item $2$
			\inlineitem $3$
			\inlineitem $1$
			\inlineitem Not enough info given
		\end{enumerate}
		%Problem 10: Chemical Reactions
		\item[10)]
		Kevin has lump of francium and a jar of water. Out of boredom, he decided to place the lump of francium into the jar of water. He realizes that after all of the francium has reacted, he was left with francium hydroxide and hydrogen gas. Can you help Kevin figure out what is the product side of the balanced molecular equation of the reaction?
		\begin{enumerate}
			\item $2FrOh+H_2$
			\inlineitem $FrOH+H_2$
			\inlineitem $2FrOH+2H$
			\inlineitem $FrOH + 2H_2$
		\end{enumerate}
		%Problem 11: Activity Series
		\item[11)]
		Elisha has a lot of different metal and liquids. She decided to mix up different metals with different liquids. Which of the following reactions would be possible?
		\begin{enumerate}
			\item Potassium and Copper II Nitrate
			\inlineitem Copper and Aluminium Nitrate
			\inlineitem Gold and Copper II Nitrate
			\inlineitem Potassium and Sodium Nitrate
		\end{enumerate}
	\end{enumerate}
	\textbf{Use the following information for Questions \#12 and \#13: } One day while doing nothing at a SciOly study session, as always, Kevin remembered his bucket of sucrose. He realized that he needed a way to get rid of it so he decided to burn the sucrose. By using a blowtorch to set the sucrose on fire, Kevin lets it completely burn away.
	\begin{enumerate}
		%Problem 12: Types of Reactions
		\item[12)]
		Can you help Kevin identify what kind of reaction is the burning of sucrose?
		\begin{enumerate}
			\item Double Replacement
			\inlineitem Stable
			\inlineitem Single Replacement
			\inlineitem Reduction and Oxidation
		\end{enumerate}
		%Problem 13: Stoichiometry
		\item[13)]
		Knowing the unbalanced molecular equation of the reaction is the following, can you help Kevin figure out the sum of the missing coefficients? \\
		\begin{center}
			$8C_{12}H_{22}O_{12} + \_\_O_2 = \_\_CO_2 + \_\_H_2O$
		\end{center}
		\begin{enumerate}
			\item 284
			\inlineitem 276
			\inlineitem 312
			\inlineitem 262
		\end{enumerate}
		%Problem 14: History
		\item[14)]
		The alchemists of the old has always been trying to make gold out of metals. Elisha was walking down the street one day when she accidentally found the Philosopher's Stone. With this, she was able to turn everything metal into gold. Which of the following particles was discovered using the element gold in some form?
		\begin{enumerate}
			\item $\alpha$
			\inlineitem $\beta$
			\inlineitem $\gamma$
			\inlineitem A proton
		\end{enumerate}
		%Problem 15: Orbital Noation
		\item[15)]
		Kevin is pretty confused about the rules of orbital notation. Can you help him determine which of the following rules is correct for orbital notation?
			\begin{enumerate}
				\item[I)] Each orbital must be filled with different spins
				\item[II)] Within a single subshell, an orbital must be completely full before moving to the next orbital
				\item[III)] Each subshell has $2n+1$ orbitals
				\item[IV)] There can be only two electrons in each orbital
			\end{enumerate}
		\begin{enumerate}
			\item I only
			\inlineitem I and IV
			\inlineitem I and II
			\inlineitem III and IV
		\end{enumerate}
		%Problem 16: Atomic Structure
		\item[16)]
		Kevin is getting pretty lazy at writing questions. Here is just a normal problem. $>$ Implying the other problems weren't normal enough. In a cation, which is the comparision between the number of electrons($e$) and protons($p$)?
		\begin{enumerate}
			\item $e<p$
			\inlineitem $e=p$
			\inlineitem $e<p$
			\inlineitem Cannot be determined
		\end{enumerate}
		%Problem 17: Weight Percentage
		\item[17)]
		Elisha was testing random compounds that she found in various different chemistry rooms. For one of the compounds, she realizes it was $36\%$ carbon, $32\%$ oxygen, and $32\%$ helium. Don't worry if this isn't a real compound but what is the empirical equation of the compound?
		\begin{enumerate}
			\item $C_3O_2H_8$
			\inlineitem $C_4O_3H_8$
			\inlineitem $C_3OH_8$
			\inlineitem $C_1O_1H_2$
		\end{enumerate}
		%Problem 18: Stoich
		\item[18)]
		Did you know that farts are made of methane? At least thats what Kevin thinks. By using a super high tech laser set up, Kevin was able to collect $120\;g$ of methane gas($CH_4$). If Kevin were to burn his collection of methane gas into carbon dioxide and water vapor, how many grams of $CO_2$ would he end up with.
		\begin{enumerate}
			\item $7.48$
			\inlineitem $329.25$
			\inlineitem $215.34$
			\inlineitem $440.1$
		\end{enumerate}
		%Problem 19: Chemical Reactions
		\item[19)]
		Kevin almost did something extremely bad. He almost had the thought of just copying the previous year's test questions. But due to his high sense of honor and pride, Kevin decided to ask you the following questions. Which one of the elements in the answer below is most reactive with water.
		\begin{enumerate}
			\item $K$
			\inlineitem $Mg$
			\inlineitem $Mn$
			\inlineitem $Ge$
		\end{enumerate}
		%Problem 20: Types of Reactions
		\item[20)]
		Man, Kevin has really been using himself alot in these questions. That's kinda really bad because when he started making these tests, he told himself he would use the two captains equally. To make up for this, Kevin decided perform a chemical reaction.(Don't ask why) The chemical equation is:
			\begin{center}
				$Mg + CuSO_4\rightarrow Cu + MgSO_4$
			\end{center}
		What type of reaction is this?
		\begin{enumerate}
			\item Double replacement
			\inlineitem Reduction and Oxidation
			\inlineitem Single Replacement
			\inlineitem Combustion
		\end{enumerate}
		%Problem 21: Stoich
		\item[21)]
		Elisha was able to extract some potassium from a banana. She consequently mixed the potassium with a solution of copper II sulfate. The potassium will react with the copper II sulfate in a single replacement reaction. What is the sum of the coefficients, when all coefficients are reduced to the simpliest whole number?
		\begin{enumerate}
			\item $2$
			\inlineitem $3$
			\inlineitem $4$
			\inlineitem $5$
		\end{enumerate}
		%Problem 22: Electronic Config
		\item[22)]
		Kevin has an element with an electronic configuration of $1s^22s^22p^63s^23p^64s^23d^{10}4p^65s^24d^2$. What type of element is it?
		\begin{enumerate}
			\item Nonmetal
			\inlineitem Transition Metal
			\inlineitem Metalloid
			\inlineitem Alkali Metal
		\end{enumerate}
		%Problem 23: Periodic Behavior
		\item[23)]
		Okay I have seriously ran out of creativeness. No jokes anymore. Just facts. For ionization energy, if one were to go down a column, what would happen?
		\begin{enumerate}
			\item Decrease
			\inlineitem Increase
			\inlineitem Stay the same
			\inlineitem Cannot be determined
		\end{enumerate}
		%Problem 24: Specific Heat
		\item[24)]
		Suppose that Kevin has a cup of ice of $10\;g$ at $-3^oC$. He wants to heat it up to $25^oC$. How much energy would Kevin have to put into the ice in order to raise it up to the desired temperature? The specific heat of water is $4.184\;\frac{J}{g*C}$, the specific heat of ice is $2.108\frac{J}{g*C}$, the heat of fusion for water is $333.55\;\frac{J}{g}$, and the heat of vaporization for water is $2257\;\frac{J}{g}$.
		\begin{enumerate}
			\item $1109.24$
			\inlineitem $23679.24$
			\inlineitem $4444.74$
			\inlineitem $5564.56$
		\end{enumerate}
		%Problem 25: Graphs of Phase Change
		\item[25)]
		We know that ice floats on water. What would be the slope of the solid/liquid phase change line?
		\begin{enumerate}
			\item Positive
			\inlineitem Zero
			\inlineitem Infinity/DNE
			\inlineitem Negative
		\end{enumerate}
	\end{enumerate}
\end{document}