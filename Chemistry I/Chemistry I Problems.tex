\documentclass{article}
\usepackage{amsmath}
\usepackage{framed}
\usepackage[inline]{enumitem}  
\usepackage[pdftex]{graphicx}
\usepackage[letterpaper, total={7.5in, 10in}]{geometry}
\title{JPS Science League: Chemistry I}
\author{Kevin Yang and Elisha Zhao}
\date{Entrance Exam}
\makeatletter
% This command ignores the optional argument for itemize and enumerate lists
\newcommand{\inlineitem}[1][]{%
\ifnum\enit@type=\tw@
    {\descriptionlabel{#1}}
  \hspace{\labelsep}%
\else
  \ifnum\enit@type=\z@
       \refstepcounter{\@listctr}\fi
    \quad\@itemlabel\hspace{\labelsep}%
\fi}
\makeatother
\parindent=0pt
\begin{document}
	\maketitle
	\renewcommand{\labelenumii}{\alph{enumii})}
	\textbf{Instructions: }There are $25$ test questions and $5$ bonus questions in this exam. The $25$ questions will determine your placement while the $5$ bonus questions will serve as tie breakers. You will be given $50$ minutes for this exam. Points will not be taken off for wrong answers so you are encouraged to answer every question. Remember, finish the $25$ questions before starting the bonus. Good luck and have fun!
	\begin{enumerate}
		%Question 1: Measurement
		\item[1)]
		Something really small can be very important. The prefix "pico" is ued to indicate something very small. What power of $10$ is equivalent to "pico".
		\begin{enumerate}
			\item $-3$
			\inlineitem $-9$
			\inlineitem $-12$
			\inlineitem $-15$
		\end{enumerate}
		%Question 2: Properties
		\item[2)]
		
	l\end{enumerate}
\end{document}