\documentclass{article}
\usepackage{amsmath}
\usepackage{framed}
\usepackage[inline]{enumitem}  
\usepackage[pdftex]{graphicx}
\usepackage[letterpaper, total={7.5in, 10in}]{geometry}
\title{JPS Science League: Chemistry I}
\author{Kevin Yang and Elisha Zhao}
\date{Entrance Exam}
\makeatletter
% This command ignores the optional argument for itemize and enumerate lists
\newcommand{\inlineitem}[1][]{%
\ifnum\enit@type=\tw@
    {\descriptionlabel{#1}}
  \hspace{\labelsep}%
\else
  \ifnum\enit@type=\z@
       \refstepcounter{\@listctr}\fi
    \quad\@itemlabel\hspace{\labelsep}%
\fi}
\makeatother
\parindent=0pt
\begin{document}
	\maketitle
	\renewcommand{\labelenumii}{\alph{enumii})}
	\textbf{Instructions: }There are $25$ test questions and $5$ bonus questions in this exam. The $25$ questions will determine your placement while the $5$ bonus questions will serve as tie breakers. You will be given $50$ minutes for this exam. Points will not be taken off for wrong answers so you are encouraged to answer every question. Remember, finish the $25$ questions before starting the bonus. Good luck and have fun!
	\begin{enumerate}
		%Question 1: Measurement
		\item[1)]
		Something really small can be very important. The prefix "pico" is ued to indicate something very small. What power of $10$ is equivalent to "pico".
		\begin{enumerate}
			\item $-3$
			\inlineitem $-9$
			\inlineitem $-12$
			\inlineitem $-15$
		\end{enumerate}
		%Question 2: Properties
		\item[2)]
		Kevin found the following statements on a piece of crumpled paper lying on the ground.
		\begin{enumerate}
			\item[A.] Mass is conserved
			\item[B.] Moles are conserved
			\item[C.] Volume is conserved
			\item[D.] Moles are conserved
		\end{enumerate}
		Which is (are) always true for a chemical reaction?
		\begin{enumerate}
			\item 
			\inlineitem 
			\inlineitem 
			\inlineitem 
		\end{enumerate}
		%Question 3: Compounds
		\item[3)]
		Elisha was studying Chem Lab for Science Olympiad by attempting to match different compounds with their names. Can you help her figure out which compound has the correct name paired with it?
		\begin{enumerate}
			\item Sodium Dioxide and $NO_2$
			\inlineitem 
			\inlineitem 
			\inlineitem 
		\end{enumerate}
		%Problem 4: Formulas
		\item[4)]
		Kevin often gets hungry at midnight. Often times, he goes to the fridge to find something tasty to eat. 2 days ago, he picked up some calcium nitrate for a small snacc. Can you help him figure out what the chemical formula for calcium nitrate?
		\begin{enumerate}
			\item 
			\inlineitem 
			\inlineitem 
			\inlineitem 
		\end{enumerate}
		%Problem 5: Formulas
		\item[5)]
		After eating some calcium nitrate, Kevin thought it wasn't very tasty so he decided to try out sodium sulfate. Can you figure out the mass ratio of sodium and sulfur in sodium sulfate?
		\begin{enumerate}
			\item 
			\inlineitem 
			\inlineitem 
			\inlineitem 
		\end{enumerate}
		%Problem 6: Mole
		\item[6)]
		Elisha found a bucket of glucose sugar in Mr. D room. (Don't ask why that bucket exists) Can you find out how many moles of H would you find in $5.2\; kg$ of $C_6H_{12}O_6$.
		\begin{enumerate}
			\item 
			\inlineitem 
			\inlineitem 
			\inlineitem 
		\end{enumerate}
		%Problem 7: Weight Percent
		\item[7)]
		It turns out that Kevin placed that bucket in Mr. D's room. We don't actually know why he did it but after we asked him, he informed us that he also made a bucket of sucrose($C^{12}H_{22}O_{12}$). Kevin wasn't exactly sure how much sucrose he put in the bucket but he was sure that he put $3.6\;kg$ of carbon atoms. Can you help him figure out of much sucrose he has?
		\begin{enumerate}
			\item 
			\inlineitem 
			\inlineitem 
			\inlineitem 
		\end{enumerate}
		%Problem 8: Properties
		\item[8)]
		The US National Chemistry Olympiad is something extremely fun but not exactly easy. While taking the USNCO, Kevin had to perform an experiment to determine the molarity of acid in a beaker. To do so, he decided to titrate 5 individual samples of acid from the beaker. As a result he found the molarity is: $4.53 \pm 0.002$, $2.45 \pm 0.005$, $8.43 \pm 0.001$, $5.23\pm 0.002$, and $6.45\pm 0.001$. Is Kevin accurate and/or precise?
		\begin{enumerate}
			\item 
			\inlineitem 
			\inlineitem 
			\inlineitem 
		\end{enumerate}
		%Problem 9: Compounds
		\item[9)]
		Every day when you're walking down the street, everybody that you meet has an original point of view. And I say - Hey! What a wonderful kind of day! If we could learn to work and play and get along with each other to analyze some compounds with polyatomic ions. One day when Elisha was walking down the street, she found some barium dicrhomate $BaCr_2O_7$. Can you help her figure out how many sodium ions would be in sodium dichromate?
		\begin{enumerate}
			\item 
			\inlineitem 
			\inlineitem 
			\inlineitem 
		\end{enumerate}
		%Problem 10: Chemical Reactions
		\item[10)]
		Kevin has lump of francium and a jar of water. Out of boredom, he decided to place the lump of francium into the jar of water. He realizes that after all of the francium has reacted, he was left with francium hydroxide and hydrogen gas. Can you help Kevin figure out what is the balanced molecular equation of the reaction?
		\begin{enumerate}
			\item 
			\inlineitem 
			\inlineitem 
			\inlineitem 
		\end{enumerate}
		%Problem 11: Activity Series
		\item[11)]
		Elisha has a lot of different metal and liquids. She decided to mix up different metals with different liquids. Which of the following reactions would be possible?
		\begin{enumerate}
			\item 
			\inlineitem 
			\inlineitem 
			\inlineitem 
		\end{enumerate}
	\end{enumerate}
	\textbf{Use the following information for Questions \#12 and \#13: } One day while doing nothing at a SciOly study session, as always, Kevin remembered his bucket of sucrose. He realized that he needed a way to get rid of it so he decided to burn the sucrose. By using a blowtorch to set the sucrose on fire, Kevin lets it completely burn away.
	\begin{enumerate}
		%Problem 12: Types of Reactions
		\item[12)]
		Can you help Kevin identify what kind of reaction is the burning of sucrose?
		\begin{enumerate}
			\item 
			\inlineitem 
			\inlineitem 
			\inlineitem 
		\end{enumerate}
		%Problem 13: Stoichiometry
		\item[13)]
		Knowing the unbalanced molecular equation of the reaction is the following, can you help Kevin figure out the sum of the missing coefficients? \\
		\begin{centered}
			$8C_{12}H_{22}O_{12} + __O_2 = __ $
		\end{centered}
		\begin{enumerate}
			\item 
			\inlineitem 
			\inlineitem 
			\inlineitem 
		\end{enumerate}
	\end{enumerate}
\end{document}